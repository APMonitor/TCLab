\documentclass[tikz]{standalone}
\usepackage{units}
\usepackage{helvet}
\usepackage[T1]{sansmath}
\renewcommand{\familydefault}{\sfdefault}
\normalfont

\usetikzlibrary{arrows,positioning} 

\begin{document}
\begin{sansmath}
 
\tikzstyle{block} = [
    draw, 
    minimum height = 1.0cm,
    minimum width = 5.0cm,
    rounded corners,
    fill=yellow!20,
    rectangle,
    text centered]

\tikzstyle{env} = [
    draw, 
    minimum height = 4.0cm,
    minimum width = 6.0cm,
    rounded corners,
    fill=blue!20,
    rectangle,
    text centered]

\begin{tikzpicture}[auto,thick,node distance = 1cm]
    %\draw[step=1cm,gray,very thin] (0,0) grid (16,9);
  
    \draw (5,7) node[env] (laptop) {};
    \draw (5,2) node[env] (device) {};
    \draw (5,8) node[block] (top) {\large\bf Jupyter notebooks};
    \draw (5,6) node[block] (mid) {\large\bf tclab.py};
    \draw (5,3) node[block] (bot) {\large\bf sketch.ino};
    \draw (5,1) node[block] (ard) {\large\bf Arduino};

    \draw[->, ultra thick] (top) ++(-1,-0.5) -- ++(0,-1);
    \draw[->, ultra thick] (mid) ++(-1,-0.5) -- ++(0,-2);
    \draw[->, ultra thick] (bot) ++(-1,-0.5) -- ++(0,-1);

    \draw[<-, ultra thick] (top) ++(1,-0.5) -- ++(0,-1);
    \draw[<-, ultra thick] (mid) ++(1,-0.5) -- ++(0,-2);
    \draw[<-, ultra thick] (bot) ++(1,-0.5) -- ++(0,-1);

    \draw (7,4.5) node {USB};
    \draw (2.5,8.75) node[right] {student laptop};
    \draw (2.5,0.25) node[right] {Arduino + tclab};	

\end{tikzpicture}
\end{sansmath}
\end{document}